\documentclass[a4paper, 10pt]{article}
\usepackage[left=1.5cm,text={18cm, 25cm},top=2.5cm]{geometry}
\usepackage[utf8]{inputenc}
\usepackage[czech]{babel}
\usepackage{setspace}
\usepackage{times}
\usepackage{sectsty}

\sectionfont{\fontsize{12}{15}\bfseries}
\subsectionfont{\fontsize{11}{15}\bfseries}


\begin{document}
    \noindent
    Dokumentace úlohy MKA: Minimalizace konečného automatu v Python 3.6 do IPP 2016/2017 \\*
    Jméno a příjmení: Jiří Matějka \\*
    Login: xmatej52

    \section{Zadání, použité funkce a třídy}
        Úkolem bylo vytvořit script, který provede minimalizaci zadaného konečného automatu,
        včetně kontroly, zda zadaný automat odpovídá definici dobře specifikovaného
        konečného automatu.
        Pro tvorbu scriptu byly hojně využity abstraktní datové typy, které python
        nabízí, zejmená řetězec, slovník, seznam a množina. Dále byla použita knihovna
        re pro kontrolu, zda jméno stavu konečného automatu je validní.
    \section{Postup řešení}
        \subsection{Zpracování parametrů programu}
            Parametry (argumenty) programu jsou zpracovány pomocí třídy Arguments.
            Objekt této třídy uchovává veškěré nastavení programu. Construktor třídy
            zároveň kontroluje správnost argumentů.
        \subsection{Načítání vstupních dat}
            Při načítání vstupního soubory byly odstraněny veškeré bílé znaky a
            komentáře. Následně byl takto načtený řetězec zpracován pomocí rozsáhlého
            stavového automatu, během kterého probíhala kontrola syntaktických a
            sémantických chyb.
        \subsection{Zpracování načtených vstupních dat}
            Zpracování načtených dat byla nejnáročnější část úlohy a zpracovávala
            se pomocí metod třídy FiniteStateMachine, která sloužila i k reprezentaci
            konečného automatu. Nejdříve byly
            nalezeny dostupné a nedostupné stavy, následně neukončující stavy, poté
            nedeterminismy a následně epsilon přechody. Pokud kontrola, zda se jedná
            o dobře specifikovaný konečný automat proběhla v pořádku, mohlo dojít buď
            k výpisu výsledků (Pokud nebyla požadována minimalizace). Samotná tvorba
            algoritmu pro minimilizaci zabrala nejvíce času. Metoda pro minimalizaci
            obsahuje 2 pomocné funkce a 5 -- 6 do sebe vnořených cyklů. Algoritmus
            je založen na algoritmu probíraném v rámci předmětu IFJ.
        \subsection{Vytvoření výstupního souboru}
            Výstupní soubor se otvírá až poté, co proběhla kontrola syntaktických
            a sémantických chyb a nepředpokládá se, že by během programu mohla nastat
            chyba.
        \subsection{Řešení chybových stavů}
            Chyby otevření souborů jsem češil pomocí odchytnutí výjimky. Ostatní
            chyby jsem detekoval pomocí podmíněných skoků. Chyby zpracovávala
            vlastní funkce, která vytiskla požadovanou zprávu na chybový výstup
            a ukončila program se zadaným chybovým kódem.
        \subsection{Testování}
            Během tvorby scriptu jsem script testoval na svém stroji, (Ubuntu 16.04)
            a to interpretem Python 3.5. Poté, co mi script pracoval na mém stroji,
            jsem testování spustil na školním serveru Merlin, kde proběhlo, až na
            jeden referenční test (chybějící odřádkování), v pořádku.
        \subsection{Odevzdání}
            Úlohu jsem se snažil dokončit tak, abych stihl pokusné odevzdání.
            Až na pár drobných úprav se mi to podařilo i s rezervou jednoho dne.
%            a hodnocení programu bylo v rozmezí od 80\% -- 100\%.
            Dokumentaci jsem tvořil před pokusném odevzdání (po pokusném odevzdání
            probělo pár drobných úprav) a to pomocí \LaTeX u. Při dokumentaci
            kódu jsem se snažil používat dokumentační řetězce ke všem funkcím a
            třídám a snažil jsem se přehledně komentovat svůj kód, aby byl co
            nejlépe čitelný a to nejen pro mě. Dokumentace kódu je v anglickém
            jazyce.

    \section{Shrnutí z pohledu autora}
        Tento projekt nebyla má první zkušenost s jazykem Python a už jsem byl
        i dokonce zvyklý na syntax a měl jsem alespoň minimální vědomí, jak
        interpret Pythonu funguje. Zkušenosti s Pythonem jsem měl v rámci předmětu
        scriptovací jazyky a spolupráci s výzkumnou skupinou KNOT. Nicméně přesto
        Python není u mě nijak oblíbený jazyk, zejména kvůli syntaxy. Programy v
        pythonu mi osobně přijdou značně nečitelné, zejména, když odsazení
        od začátku stránky je už příliš velké.
\end{document}